% Exercitium2018.tex 
% Exercitium (curso 2018-19)
% José A. Alonso Jiménez
% Sevilla, 23 de marzo de 2019
% ======================================================================

\documentclass[a4paper,12pt,twoside]{book}

%%%%%%%%%%%%%%%%%%%%%%%%%%%%%%%%%%%%%%%%%%%%%%%%%%%%%%%%%%%%%%%%%%%%%%%%
%% § Paquetes adicionales
%%%%%%%%%%%%%%%%%%%%%%%%%%%%%%%%%%%%%%%%%%%%%%%%%%%%%%%%%%%%%%%%%%%%%%%%

% Configuración para XeLaTeX
\usepackage{fontspec}
\usepackage{xltxtra}
\defaultfontfeatures{Ligatures=TeX,Numbers=OldStyle}
\setromanfont{DejaVu Sans}
% \setsansfont{Arial}
\setmonofont{DejaVu Sans Mono}[Scale={0.90}]

% Notas: La lista de fuentes disponibles se obtiene con fc-list

% \usepackage{ucs}
% \usepackage[utf8]{inputenc}        % Acentos de UTF8
\usepackage[spanish]{babel}        % Castellanización.
% \usepackage[T1]{fontenc}           % Codificación T1 con European Computer
%                                    % Modern.  
% \usepackage{graphicx}
% \usepackage{fancyvrb}              % Verbatim extendido
% \usepackage{mathpazo}              % Fuentes semejante a palatino
% \usepackage[scaled=.90]{helvet}
% \usepackage{cmtt}
% \renewcommand{\ttdefault}{cmtt}
\usepackage{a4wide}
\usepackage{minted}
\usepackage{comment}
\usepackage{amssymb, amsmath, amsbsy}

\usepackage{titletoc}
\dottedcontents{chapter}[0em]{}{32em}{1pc}
% \dottedcontents{chapter}[<left>]{<above-code>}{<label width>}{<leader width>}

\linespread{1.05}                  % Distancia entre líneas
\setlength{\parindent}{2em}        % Indentación de comienzo de párrafo
\setlength{\parskip}{1ex}          % Distancia entre párrafos
% \deactivatetilden                  % Elima uso de ~ para la eñe
\raggedbottom                      % No ajusta los espacios verticales.

\usepackage[%
  colorlinks=true,
  urlcolor=blue,
  % pdftex,
  pdfauthor={José A. Alonso <jalonso@us.es>},%
  pdftitle={Exercitium (curso 2018-19)},%
  pdfstartview=FitH,%
  bookmarks=false]{hyperref}      

\setcounter{tocdepth}{1}
\setcounter{secnumdepth}{4}

\usepackage{tocstyle}
\usetocstyle{KOMAlike}

% \usepackage{tocloft}
% \renewcommand\cftpartnumwidth{3cm}

% \usepackage{minitoc}

% \setlength\cftparskip{-2pt}
% \setlength\cftbeforechapskip{0pt}

%%%%%%%%%%%%%%%%%%%%%%%%%%%%%%%%%%%%%%%%%%%%%%%%%%%%%%%%%%%%%%%%%%%%%%%%
%% § epigraph: Para citas al principio del capítulo                   %%
%%%%%%%%%%%%%%%%%%%%%%%%%%%%%%%%%%%%%%%%%%%%%%%%%%%%%%%%%%%%%%%%%%%%%%%%

% \usepackage{epigraph} % Para citas al principio del capítulo
\usepackage{epigraph,varwidth}

\renewcommand{\epigraphsize}{\small}
\setlength{\epigraphwidth}{0.6\textwidth}
\renewcommand{\textflush}{flushleft}
\renewcommand{\sourceflush}{flushleft}
% A useful addition
\newcommand{\epitextfont}{\itshape}
% \newcommand{\episourcefont}{\scshape}
\newcommand{\episourcefont}{\normalfont}

\makeatletter
\newsavebox{\epi@textbox}
\newsavebox{\epi@sourcebox}
\newlength\epi@finalwidth
\renewcommand{\epigraph}[2]{%
  \vspace{\beforeepigraphskip}
  {\epigraphsize\begin{\epigraphflush}
   \epi@finalwidth=\z@
   \sbox\epi@textbox{%
     \varwidth{\epigraphwidth}
     \begin{\textflush}\epitextfont#1\end{\textflush}
     \endvarwidth
   }%
   \epi@finalwidth=\wd\epi@textbox
   \sbox\epi@sourcebox{%
     \varwidth{\epigraphwidth}
     \begin{\sourceflush}\episourcefont#2\end{\sourceflush}%
     \endvarwidth
   }%
   \ifdim\wd\epi@sourcebox>\epi@finalwidth 
     \epi@finalwidth=\wd\epi@sourcebox
   \fi
   \leavevmode\vbox{
     \hb@xt@\epi@finalwidth{\hfil\box\epi@textbox}
     \vskip1.75ex
     \hrule height \epigraphrule
     \vskip.75ex
     \hb@xt@\epi@finalwidth{\hfil\box\epi@sourcebox}
   }%
   \end{\epigraphflush}
   \vspace{\afterepigraphskip}}}
\makeatother

%%%%%%%%%%%%%%%%%%%%%%%%%%%%%%%%%%%%%%%%%%%%%%%%%%%%%%%%%%%%%%%%%%%%%%%%%%%%%%
%% § Cabeceras                                                              %%
%%%%%%%%%%%%%%%%%%%%%%%%%%%%%%%%%%%%%%%%%%%%%%%%%%%%%%%%%%%%%%%%%%%%%%%%%%%%%%

\usepackage{fancyhdr}

\addtolength{\headheight}{\baselineskip}

\pagestyle{fancy}

\cfoot{}

\fancyhead{}
\fancyhead[RE]{\mdseries\sffamily Exercitium (2018--19)}
\fancyhead[LO]{\mdseries\sffamily \nouppercase{\leftmark}}
\fancyhead[LE,RO]{\mdseries\sffamily \thepage}

%%%%%%%%%%%%%%%%%%%%%%%%%%%%%%%%%%%%%%%%%%%%%%%%%%%%%%%%%%%%%%%%%%%%%%%%
%% § Definiciones
%%%%%%%%%%%%%%%%%%%%%%%%%%%%%%%%%%%%%%%%%%%%%%%%%%%%%%%%%%%%%%%%%%%%%%%%

\input definiciones
\def\mtctitle{Contenido}

%%%%%%%%%%%%%%%%%%%%%%%%%%%%%%%%%%%%%%%%%%%%%%%%%%%%%%%%%%%%%%%%%%%%%%%%
%% § Título
%%%%%%%%%%%%%%%%%%%%%%%%%%%%%%%%%%%%%%%%%%%%%%%%%%%%%%%%%%%%%%%%%%%%%%%%

\title{
  {\LARGE Exercitium (curso 2018--19) \\
  {\Large Ejercicios de programación funcional con Haskell \\
  {\normalsize (hasta el 15 de marzo de 2019)}}} }  
\author{\href{http://www.cs.us.es/~jalonso}
        {\Large José A. Alonso Jiménez}}
\date{\vfill \hrule \vspace*{2mm}
  \begin{tabular}{l}
      \href{http://www.cs.us.es/glc}
           {Grupo de Lógica Computacional} \\
      \href{http://www.cs.us.es}
           {Dpto. de Ciencias de la Computación e Inteligencia Artificial} \\
      \href{http://www.us.es}
           {Universidad de Sevilla}  \\
      Sevilla, \today 
  \end{tabular}\hfill\mbox{}}

%%%%%%%%%%%%%%%%%%%%%%%%%%%%%%%%%%%%%%%%%%%%%%%%%%%%%%%%%%%%%%%%%%%%%%%%
%% § Documento
%%%%%%%%%%%%%%%%%%%%%%%%%%%%%%%%%%%%%%%%%%%%%%%%%%%%%%%%%%%%%%%%%%%%%%%%

% \includeonly{Números con dígitos 1 y 2}

% \includexmp{licencia}

\begin{document}
% \dominitoc

\maketitle
\newpage

\input{licenciaCC}
\newpage

\newpage

\mbox{} \vspace*{2cm}
  \begin{verse}
  ``Sorpresas tiene la vida, \\
  Guiomar, del alma y del cuerpo; \\ 
  que nadie guarde hasta el fin \\
  el nombre que le pusieron; \\
  nadie crea ser quien dicen \\
  que es, ni que pueda serlo.'' \\ \vspace*{2ex}

  De Antonio Machado
  \end{verse}

\begin{flushright} 
\textit{Para Guiomar}
\end{flushright}

\newpage

\tableofcontents
\clearpage

\renewcommand{\chaptername}{Ejercicio}

\chapter*{Introducción}

% \mbox{} \hspace*{1cm} 

\begin{quote}
  ``\textit{The chief goal of my work as an educator and author is to
  help people learn to write beautiful programs.}''

  (Donald Knuth en
  \href{http://www.paulgraham.com/knuth.html}{Computer programming as an art})
\end{quote}

\vspace* {1cm}

Este libro es una recopilación de las soluciones de los ejercicios
propuestos en el blog
\href{https://www.glc.us.es/~jalonso/exercitium}
     {Exercitium}\
     \footnote{\url{https://www.glc.us.es/~jalonso/exercitium}}
durante el curso 2018--19.

El principal objetivo de Exercitium es servir de complemento a la
asignatura de
\href{https://www.cs.us.es/~jalonso/cursos/i1m-18}
     {Informática}\
     \footnote{\url{https://www.cs.us.es/~jalonso/cursos/i1m-18}}
de 1º del Grado en Matemáticas de la Universidad de Sevilla.

Con los problemas de Exercitium, a diferencias de los de las
\href{https://www.cs.us.es/~jalonso/cursos/i1m-18/ejercicios/ejercicios-I1M-2018.pdf}
     {relaciones}\
     \footnote{\url{https://www.cs.us.es/~jalonso/cursos/i1m-18/ejercicios/ejercicios-I1M-2018.pdf}},
se pretende practicar con los conocimientos adquiridos durante todo el
curso, mientras que con las relaciones están orientadas a los nuevos
conocimientos.

Habitualmente de cada ejercicio se muestra distintas soluciones y se
compara sus eficiencias.

La dinámica del blog es la siguiente: cada día, de lunes a viernes, se
propone un ejercicio para que los alumnos escriban distintas soluciones
en los comentarios. Pasado 7 días de la propuesta de cada ejercicio, se
cierra los comentarios y se publica una selección de sus soluciones.

Para conocer la cronología de los temas explicados se puede consultar el
\href{https://www.glc.us.es/~jalonso/vestigium/category/curso/i1m/i1m2018}
     {diario de clase}\
     \footnote{\url{https://www.glc.us.es/~jalonso/vestigium/category/curso/i1m/i1m2018}}.

En el libro se irán añadiendo semanalmente las soluciones de los
ejercicios del curso.

El código del libro se encuentra en
\href{https://github.com/jaalonso/Exercitium2018}
     {GitHub}\
     \footnote{\url{https://github.com/jaalonso/Exercitium2018}}

\section*{Cuaderno de bitácora}

En esta sección se registran los cambios realizados en las sucesivas
versiones del libro.

\subsection*{Versión del 16 de diciembre de 2018}

Se han añadido los ejercicios resueltos de la primera semana de
diciembre:

\begin{itemize}
\item \nameref{031218}
\item \nameref{041218}
\item \nameref{051218}
\item \nameref{061218}
\item \nameref{071218}
\end{itemize}
     
\subsection*{Versión del 22 de diciembre de 2018}

Se han añadido los ejercicios resueltos de la primera semana de
diciembre:

\begin{itemize}
\item \nameref{101218}
\item \nameref{111218}
\item \nameref{121218}
\item \nameref{131218}
\item \nameref{141218}
\end{itemize}
     
\subsection*{Versión del 29 de diciembre de 2018}

Se han añadido los ejercicios resueltos de la primera semana de
diciembre:

\begin{itemize}
\item \nameref{171218}
\item \nameref{181218}
\item \nameref{191218}
\item \nameref{201218}
\item \nameref{211218}
\end{itemize}
     
\subsection*{Versión del 29 de diciembre de 2018}

Se han añadido los ejercicios resueltos del 24 al 28 de
diciembre:

\begin{itemize}
\item \nameref{181224}
\item \nameref{181225}
\item \nameref{181226}
\item \nameref{181227}
\item \nameref{181228}
\end{itemize}

\subsection*{Versión del 12 de enero de 2019}

Se han añadido los ejercicios resueltos del 31 de diciembre al 4 de
enero: 

\begin{itemize}
\item \nameref{181231}
\item \nameref{190101}
\item \nameref{190102}
\item \nameref{190103}
\item \nameref{190104}
\end{itemize}

\subsection*{Versión del 19 de enero de 2019}

Se han añadido los ejercicios resueltos hasta el 11 de enero:

\begin{itemize}
\item \nameref{190107}
\item \nameref{190108}
\item \nameref{190109}
\item \nameref{190110}
\item \nameref{190111}
\end{itemize}

\subsection*{Versión del 26 de enero de 2019}

Se han añadido los ejercicios resueltos hasta del 14 al 18 de enero:

\begin{itemize}
\item \nameref{190114}
\item \nameref{190115}
\item \nameref{190116}
\item \nameref{190117}
\item \nameref{190118}
\end{itemize}

\subsection*{Versión del 2 de febrero de 2019}

Se han añadido los ejercicios resueltos hasta del 21 al 25 de enero:

\begin{itemize}
\item \nameref{190121}
\item \nameref{190122}
\item \nameref{190123}
\item \nameref{190124}
\item \nameref{190125}
\end{itemize}

\subsection*{Versión del 9 de febrero de 2019}

Se han añadido los ejercicios resueltos hasta del 28 de enero al 1 de
febrero:

\begin{itemize}
\item \nameref{190128}
\item \nameref{190129}
\item \nameref{190130}
\item \nameref{190131}
\item \nameref{190201}
\end{itemize}

\subsection*{Versión del 16 de febrero de 2019}

Se han añadido los ejercicios resueltos hasta del 4 al 8 de febrero:

\begin{itemize}
\item \nameref{190204}
\item \nameref{190205}
\item \nameref{190206}
\item \nameref{190207}
\item \nameref{190208}
\end{itemize}

\subsection*{Versión del 23 de febrero de 2019}

Se han añadido los ejercicios resueltos hasta del 11 al 15 de febrero:

\begin{itemize}
\item \nameref{190211}
\item \nameref{190212}
\item \nameref{190213}
\item \nameref{190214}
\item \nameref{190215}
\end{itemize}

\subsection*{Versión del 2 de marzo de 2019}

Se han añadido los ejercicios resueltos hasta del 18 al 22 de febrero:

\begin{itemize}
\item \nameref{190218}
\item \nameref{190219}
\item \nameref{190220}
\item \nameref{190221}
\item \nameref{190222}
\end{itemize}

\subsection*{Versión del 9 de marzo de 2019}

Se han añadido los ejercicios resueltos hasta del 25 de febrero al 1 de
marzo: 

\begin{itemize}
\item \nameref{190225}
\item \nameref{190226}
\item \nameref{190227}
\item \nameref{190228}
\item \nameref{190301}
\end{itemize}

\subsection*{Versión del 16 de marzo de 2019}

Se han añadido los ejercicios resueltos del 4 al 8 de marzo: 

\begin{itemize}
\item \nameref{190304}
\item \nameref{190305}
\item \nameref{190306}
\item \nameref{190307}
\item \nameref{190308}
\end{itemize}

\subsection*{Versión del 23 de marzo de 2019}

Se han añadido los ejercicios resueltos del 11 al 15 de marzo: 

\begin{itemize}
\item \nameref{190311}
\item \nameref{190312}
\item \nameref{190313}
\item \nameref{190314}
\item \nameref{190315}
\end{itemize}

\chapter{Listas equidigitales}
\entrada{Listas_equidigitales}

\chapter{Distancia de Hamming}
\entrada{Distancia_de_Hamming}

\chapter{Último dígito no nulo del factorial}
\entrada{Ultimo_digito_no_nulo_del_factorial}

\chapter{Diferencia simétrica}
\entrada{Diferencia_simetrica}

\chapter{Números libres de cuadrados}
\entrada{Numeros_libres_de_cuadrados}

\chapter{Capicúas productos de dos números de dos dígitos}
\entrada{Capicuas_productos_de_dos_numeros_de_dos_digitos}

% ---------------------------------------------------------------------
 
\chapter{Números autodescriptivos}
\entrada{Numeros_autodescriptivos}

\chapter{Número de parejas}
\entrada{Numeros_de_parejas}

\chapter{Reconocimiento de particiones}
\entrada{Reconocimiento_de_particiones}

\chapter{Relación definida por una partición}
\entrada{Relacion_definida_por_una_particion}

\chapter{Ceros finales del factorial}
\entrada{Ceros_finales_del_factorial}

% ---------------------------------------------------------------------
 
\chapter{Números primos sumas de dos primos}
\entrada{Numeros_primos_sumas_de_dos_primos}

\chapter{Suma de inversos de potencias de cuatro}
\entrada{Suma_de_inversos_de_potencias_de_cuatro}

\chapter{Elemento solitario}
\entrada{Elemento_solitario}

\chapter{Números colinas}
\entrada{Numeros_colinas}

\chapter{Raíz cúbica entera}
\entrada{Raiz_cubica_entera}

% ---------------------------------------------------------------------

\chapter{Numeración de los árboles binarios completos}
\label{031218}
\entrada{Numeracion_de_arboles_binarios_completos}

\chapter{Posiciones en árboles binarios}
\label{041218}
\entrada{Posiciones_en_arboles_binarios}

\chapter{Posiciones en árboles binarios completos}
\label{051218}
\entrada{Posiciones_en_arboles_binarios_completos}

\chapter{Elemento del árbol binario completo según su
  posición}
\label{061218}
\entrada{Elemento_del_arbol_binario_completo_segun_su_posicion}

\chapter{Aproximación entre pi y e}
\label{071218}
\entrada{Aproximacion_entre_pi_y_e}

% ---------------------------------------------------------------------

\chapter{Menor contenedor de primos}
\label{101218}
\entrada{Menor_contenedor_de_primos}

\chapter{Árbol de computación de Fibonacci}
\label{111218}
\entrada{Arbol_de_computacion_de_Fibonacci}

\chapter{Entre dos conjuntos}
\label{121218}
\entrada{Entre_dos_conjuntos}

\chapter{Expresiones aritméticas generales}
\label{131218}
\entrada{Expresiones_aritméticas_generales}

\chapter{Superación de límites}
\label{141218}
\entrada{Superacion_de_limites}

% ---------------------------------------------------------------------

\chapter{Intercambio de la primera y última columna de
  una matriz}
\label{171218}
\entrada{Intercambio_de_la_primera_y_ultima_columna_de_una_matriz}

\chapter{Números primos de Pierpont}
\label{181218}
\entrada{Numeros_primos_de_Pierpont}

\chapter{Grado exponencial}
\label{191218}
\entrada{Grado_exponencial}

\chapter{Divisores propios maximales}
\label{201218}
\entrada{Divisores_propios_maximales}

\chapter{Árbol de divisores}
\label{211218}
\entrada{Arbol_de_divisores}

% ---------------------------------------------------------------------

\chapter{Divisores compuestos}
\label{181224}
\entrada{Divisores_compuestos}

\chapter{Número de divisores compuestos}
\label{181225}
\entrada{Numero_de_divisores_compuestos}

\chapter{Tablas de operaciones binarias}
\label{181226}
\entrada{Tablas_de_operaciones_binarias}

\chapter{Reconocimiento de conmutatividad}
\label{181227}
\entrada{Reconocimiento_de_conmutatividad}

\chapter{Árbol de subconjuntos}
\label{181228}
\entrada{Arbol_de_subconjuntos}

% ---------------------------------------------------------------------

\chapter{El teorema de Navidad de Fermat}
\label{181231}
\entrada{El_teorema_de_Navidad_de_Fermat}

\chapter{El 2019 es apocalíptico}
\label{190101}
\entrada{El_2019_es_apocaliptico}

\chapter{El 2019 es malvado}
\label{190102}
\entrada{El_2019_es_malvado}

\chapter{El 2019 es semiprimo}
\label{190103}
\entrada{El_2019_es_semiprimo}

\chapter{El 2019 es un número de la suerte}
\label{190104}
\entrada{El_2019_es_un_numero_de_la_suerte}

% ---------------------------------------------------------------------

\chapter{Cadena descendiente de subnúmeros}
\label{190107}
\entrada{Cadena_descendiente_de_subnumeros}

\chapter{Mínimo producto escalar}
\label{190108}
\entrada{Minimo_producto_escalar}

\chapter{Numeración de ternas de naturales}
\label{190109}
\entrada{Numeracion_de_ternas}

\chapter{Subárboles monovalorados}
\label{190110}
\entrada{Subarboles_monovalorados}

\chapter{Mayor prefijo con suma acotada}
\label{190111}
\entrada{Mayor_prefijo_con_suma_acotada}

% ---------------------------------------------------------------------

\chapter{Ofertas 3 por 2}
\label{190114}
\entrada{Ofertas_3_por_2}

\chapter{Representación de conjuntos mediante intervalos}
\label{190115}
\entrada{Representacion_de_conjuntos_mediante_intervalos}

\chapter{Números altamente compuestos}
\label{190116}
\entrada{Numeros_altamente_compuestos}

\chapter{Posiciones del 2019 en el número pi}
\label{190117}
\entrada{Posiciones_del_2019_en_el_numero_pi}

\chapter{Mínimo número de operaciones para transformar
  un número en otro}
\label{190118}
\entrada{Minimo_numero_de_operaciones_para_transformar_un_numero_en_otro}

%-------------------------------------------------------------------------------

\chapter{Intersección de listas infinitas crecientes}
\label{190121}
\entrada{Interseccion_de_listas_infinitas_crecientes}

\chapter{Soluciones de \(x^2 = y^3 = k\)}
\label{190122}
\entrada{Soluciones_de_x2y3k}

\chapter{Sucesión triangular}
\label{190123}
\entrada{Sucesion_triangular}

\chapter{Números primos en pi}
\label{190124}
\entrada{Numeros_primos_en_pi}

\chapter{Recorrido de árboles en espiral}
\label{190125}
\entrada{Recorrido_de_arboles_en_espiral}

%-------------------------------------------------------------------------------

\chapter{Números con dígitos 1 y 2}
\label{190128}
\entrada{Numeros_con_digitos_1_y_2}

\chapter{Árboles con n elementos}
\label{190129}
\entrada{Arboles_con_n_elementos}

\chapter{Impares en filas del triángulo de Pascal}
\label{190130}
\entrada{Impares_en_filas_del_triangulo_de_Pascal}

\chapter{Triángulo de Pascal binario}
\label{190131}
\entrada{Triangulo_de_Pascal_binario}

\chapter{Dígitos en las posiciones pares de cuadrados}
\label{190201}
\entrada{Digitos_pares_de_cuadrados}

% ------------------------------------------------------------------------------

\chapter{Límites de sucesiones}
\label{190204}
\entrada{Limites_de_sucesiones}

\chapter{Medias de dígitos de pi}
\label{190205}
\entrada{Medias_de_digitos_de_pi}

\chapter{Exterior de árboles}
\label{190206}
\entrada{Exterior_de_arboles}

\chapter{Aritmética lunar}
\label{190207}
\entrada{Aritmetica_lunar}

\chapter{Término ausente en una progresión aritmética}
\label{190208}
\entrada{Termino_ausente_en_una_progresion_aritmetica}

% ---------------------------------------------------------------------

\chapter{Particiones de enteros positivos}
\label{190211}
\entrada{Particiones_de_enteros_positivos}

\chapter{Sucesión de sumas de dos números abundantes}
\label{190212}
\entrada{Sumas_de_dos_abundantes}

\chapter{Cálculo de pi mediante la serie de Nilakantha}
\label{190213}
\entrada{Calculo_de_pi_mediante_la_serie_de_Nilakantha}

\chapter{Simplificación de expresiones booleanas}
\label{190214}
\entrada{Simplificacion_de_expresiones_booleanas}

\chapter{Sucesión de Cantor de números innombrables}
\label{190215}
\entrada{Sucesion_de_Cantor_de_numeros_innombrables}

% ---------------------------------------------------------------------

\chapter{Ternas euclídeas}
\label{190218}
\entrada{Ternas_euclideas}

\chapter{Mezcla de listas}
\label{190219}
\entrada{Mezcla_de_listas}

\chapter{Mayor exponente}
\label{190220}
\entrada{Mayor_exponente}

\chapter{Número de sumandos en suma de cuadrados}
\label{190221}
\entrada{Numero_de_sumandos_en_suma_de_cuadrados}

\chapter{Divisiones del círculo}
\label{190222}
\entrada{Divisiones_del_circulo}

% ---------------------------------------------------------------------

\chapter{Inserciones en una lista de listas}
\label{190225}
\entrada{Inserciones_en_una_lista_de_listas}

\chapter{Particiones de un conjunto}
\label{190226}
\entrada{Particiones_de_un_conjunto}

\chapter{Número de particiones de un conjunto}
\label{190227}
\entrada{Numero_de_particiones_de_un_conjunto}

\chapter{Descomposiciones en sumas de cuadrados}
\label{190228}
\entrada{Descomposiciones_en_sumas_de_cuadrados}

\chapter{Número de descomposiciones en sumas de
  cuadrados}
\label{190301}
\entrada{Numero_de_descomposiciones_en_sumas_de_cuadrados}

%---------------------------------------------------------------------

\chapter{Hojas con caminos no decrecientes}
\label{190304}
\entrada{Hojas_con_caminos_no_decrecientes}

\chapter{Las torres de Hanói}
\label{190305}
\entrada{Las_torres_de_Hanoi}

\chapter{Número como suma de sus dígitos}
\label{190306}
\entrada{Numero_como_suma_de_sus_digitos}

\chapter{Suma de segmentos iniciales}
\label{190307}
\entrada{Suma_de_segmentos_iniciales}

\chapter{Camino de máxima suma en una matriz}
\label{190308}
\entrada{Camino_de_maxima_suma_en_una_matriz}

% -----------------------------------------------------------------------

\chapter{Combinaciones divisibles}
\label{190311}
\entrada{Combinaciones_divisibles}

\chapter{Números cíclopes}
\label{190312}
\entrada{Numeros_ciclopes}

\chapter{Diagonales invertidas}
\label{190313}
\entrada{Diagonales_invertidas}

\chapter{Máxima longitud de sublistas crecientes}
\label{190314}
\entrada{Maxima_longitud_de_sublistas_crecientes}

\chapter{Cambio con el menor número de monedas}
\label{190315}
\entrada{Cambio_con_el_menor_numero_de_monedas}


\end{document}

%%% Local Variables: 
%%% mode: latex
%%% TeX-master: t
%%% End: 

