% Exercitium2018.tex
% Exercitium (curso 2018-19)
% José A. Alonso Jiménez <jalonso@us.es>
% Sevilla, 8 de diciembre de 2018
% ======================================================================

\documentclass[a4paper,12pt,twoside]{book}

%%%%%%%%%%%%%%%%%%%%%%%%%%%%%%%%%%%%%%%%%%%%%%%%%%%%%%%%%%%%%%%%%%%%%%%%
%% § Paquetes adicionales
%%%%%%%%%%%%%%%%%%%%%%%%%%%%%%%%%%%%%%%%%%%%%%%%%%%%%%%%%%%%%%%%%%%%%%%%

% Configuración para XeLaTeX
\usepackage{fontspec}
\usepackage{xltxtra}
\defaultfontfeatures{Ligatures=TeX,Numbers=OldStyle}
\setromanfont{DejaVu Sans}
% \setsansfont{Arial}
\setmonofont{DejaVu Sans Mono}[Scale={0.90}]

% Notas: La lista de fuentes disponibles se obtiene con fc-list

% \usepackage{ucs}
% \usepackage[utf8]{inputenc}        % Acentos de UTF8
\usepackage[spanish]{babel}        % Castellanización.
% \usepackage[T1]{fontenc}           % Codificación T1 con European Computer
%                                    % Modern.  
% \usepackage{graphicx}
% \usepackage{fancyvrb}              % Verbatim extendido
% \usepackage{mathpazo}              % Fuentes semejante a palatino
% \usepackage[scaled=.90]{helvet}
% \usepackage{cmtt}
% \renewcommand{\ttdefault}{cmtt}
\usepackage{a4wide}
\usepackage{minted}
\usepackage{comment}

\usepackage{titletoc}
\dottedcontents{chapter}[0em]{}{12em}{1pc}
% \dottedcontents{chapter}[<left>]{<above-code>}{<label width>}{<leader width>}

\linespread{1.05}                  % Distancia entre líneas
\setlength{\parindent}{2em}        % Indentación de comienzo de párrafo
\setlength{\parskip}{1ex}          % Distancia entre párrafos
% \deactivatetilden                  % Elima uso de ~ para la eñe
\raggedbottom                      % No ajusta los espacios verticales.

\usepackage[%
  colorlinks=true,
  urlcolor=blue,
  % pdftex,
  pdfauthor={José A. Alonso <jalonso@us.es>},%
  pdftitle={Exercitium (curso 2018-19)},%
  pdfstartview=FitH,%
  bookmarks=false]{hyperref}      

\setcounter{tocdepth}{1}
\setcounter{secnumdepth}{4}

\usepackage{tocstyle}
\usetocstyle{KOMAlike}

% \usepackage{tocloft}
% \renewcommand\cftpartnumwidth{3cm}

% \usepackage{minitoc}

% \setlength\cftparskip{-2pt}
% \setlength\cftbeforechapskip{0pt}

%%%%%%%%%%%%%%%%%%%%%%%%%%%%%%%%%%%%%%%%%%%%%%%%%%%%%%%%%%%%%%%%%%%%%%%%%%%%%%
%% § Cabeceras                                                              %%
%%%%%%%%%%%%%%%%%%%%%%%%%%%%%%%%%%%%%%%%%%%%%%%%%%%%%%%%%%%%%%%%%%%%%%%%%%%%%%

\usepackage{fancyhdr}

\addtolength{\headheight}{\baselineskip}

\pagestyle{fancy}

\cfoot{}

\fancyhead{}
\fancyhead[RE]{\mdseries\sffamily Exercitium (2018--19)}
\fancyhead[LO]{\mdseries\sffamily \nouppercase{\leftmark}}
\fancyhead[LE,RO]{\mdseries\sffamily \thepage}

%%%%%%%%%%%%%%%%%%%%%%%%%%%%%%%%%%%%%%%%%%%%%%%%%%%%%%%%%%%%%%%%%%%%%%%%
%% § Definiciones
%%%%%%%%%%%%%%%%%%%%%%%%%%%%%%%%%%%%%%%%%%%%%%%%%%%%%%%%%%%%%%%%%%%%%%%%

\input definiciones
\def\mtctitle{Contenido}

%%%%%%%%%%%%%%%%%%%%%%%%%%%%%%%%%%%%%%%%%%%%%%%%%%%%%%%%%%%%%%%%%%%%%%%%
%% § Título
%%%%%%%%%%%%%%%%%%%%%%%%%%%%%%%%%%%%%%%%%%%%%%%%%%%%%%%%%%%%%%%%%%%%%%%%

\title{
  {\LARGE Exercitium (curso 2018--19) \\
  {\Large Ejercicios de programación funcional con Haskell \\
  {\normalsize (hasta el 7 de diciembre de 2018)}}} }  
\author{\href{http://www.cs.us.es/~jalonso}
        {\Large José A. Alonso Jiménez}}
\date{\vfill \hrule \vspace*{2mm}
  \begin{tabular}{l}
      \href{http://www.cs.us.es/glc}
           {Grupo de Lógica Computacional} \\
      \href{http://www.cs.us.es}
           {Dpto. de Ciencias de la Computación e Inteligencia Artificial} \\
      \href{http://www.us.es}
           {Universidad de Sevilla}  \\
      Sevilla, 7 de diciembre de 2018
  \end{tabular}\hfill\mbox{}}

%%%%%%%%%%%%%%%%%%%%%%%%%%%%%%%%%%%%%%%%%%%%%%%%%%%%%%%%%%%%%%%%%%%%%%%%
%% § Documento
%%%%%%%%%%%%%%%%%%%%%%%%%%%%%%%%%%%%%%%%%%%%%%%%%%%%%%%%%%%%%%%%%%%%%%%%

% \includeonly{recusion_sobre_numeros_naturales}

% \includexmp{licencia}

\begin{document}
% \dominitoc

\maketitle
\newpage

\input{licenciaCC}
\newpage

\newpage

\mbox{} \vspace*{2cm}
\begin{flushright}
\textit{Para Guiomar}
\end{flushright}

\newpage

\tableofcontents
\clearpage

\renewcommand{\chaptername}{Ejercicio}

\chapter*{Introducción}

% \mbox{} \hspace*{1cm} 

\begin{quote}
  ``\textit{The chief goal of my work as an educator and author is to
  help people learn to write beautiful programs.}''

  (Donald Knuth en
  \href{http://www.paulgraham.com/knuth.html}{Computer programming as an art})
\end{quote}

\vspace* {1cm}

Este libro es una recopilación de las soluciones de los ejercicios
propuestos en el blog
\href{https://www.glc.us.es/~jalonso/exercitium}
     {Exercitium}\
     \footnote{\url{https://www.glc.us.es/~jalonso/exercitium}}
durante el curso 2018--19.

El principal objetivo de Exercitium es servir de complemento a la
asignatura de
\href{https://www.cs.us.es/~jalonso/cursos/i1m-18}
     {Informática}\
     \footnote{\url{https://www.cs.us.es/~jalonso/cursos/i1m-18}}
de 1º del Grado en Matemáticas de la Universidad de Sevilla.

Con los problemas de Exercitium, a diferencias de los de las
\href{https://www.cs.us.es/~jalonso/cursos/i1m-18/ejercicios/ejercicios-I1M-2018.pdf}
     {relaciones}\
     \footnote{\url{https://www.cs.us.es/~jalonso/cursos/i1m-18/ejercicios/ejercicios-I1M-2018.pdf}},
se pretende practicar con los conocimientos adquiridos durante todo el
curso, mientras que con las relaciones están orientadas a los nuevos
conocimientos.

Habitualmente de cada ejercicio se muestra distintas soluciones y se
compara sus eficiencias.

La dinámica del blog es la siguiente: cada día, de lunes a viernes, se
propone un ejercicio para que los alumnos escriban distintas soluciones
en los comentarios. Pasado 7 días de la propuesta de cada ejercicio, se
cierra los comentarios y se publica una selección de sus soluciones.

Para conocer la cronología de los temas explicados se puede consultar el
\href{https://www.glc.us.es/~jalonso/vestigium/category/curso/i1m/i1m2018}
     {diario de clase}\
     \footnote{\url{https://www.glc.us.es/~jalonso/vestigium/category/curso/i1m/i1m2018}}.

En el libro se irán añadiendo semanalmente las soluciones de los
ejercicios del curso.

El código del libro se encuentra en
\href{https://github.com/jaalonso/Exercitium2018}
     {GitHub}\
     \footnote{\url{https://github.com/jaalonso/Exercitium2018}}

\chapter{Listas equidigitales}
\entrada{Listas_equidigitales}

\chapter{Distancia de Hamming}
\entrada{Distancia_de_Hamming}

\chapter{Último dígito no nulo del factorial}
\entrada{Ultimo_digito_no_nulo_del_factorial}

\chapter{Diferencia simétrica}
\entrada{Diferencia_simetrica}

\chapter{Números libres de cuadrados}
\entrada{Numeros_libres_de_cuadrados}

\chapter{Capicúas productos de dos números de dos dígitos}
\entrada{Capicuas_productos_de_dos_numeros_de_dos_digitos}

% ---------------------------------------------------------------------
 
\chapter{Números autodescriptivos}
\entrada{Numeros_autodescriptivos}

\chapter{Número de parejas}
\entrada{Numeros_de_parejas}

\chapter{Reconocimiento de particiones}
\entrada{Reconocimiento_de_particiones}

\chapter{Relación definida por una partición}
\entrada{Relacion_definida_por_una_particion}

\chapter{Ceros finales del factorial}
\entrada{Ceros_finales_del_factorial}

% ---------------------------------------------------------------------
 
\chapter{Números primos sumas de dos primos}
\entrada{Numeros_primos_sumas_de_dos_primos}

\chapter{Suma de inversos de potencias de cuatro}
\entrada{Suma_de_inversos_de_potencias_de_cuatro}

\chapter{Elemento solitario}
\entrada{Elemento_solitario}

\chapter{Números colinas}
\entrada{Numeros_colinas}

\chapter{Raíz cúbica entera}
\entrada{Raiz_cubica_entera}

\chapter{Numeración de los árboles binarios completos}
\entrada{Numeracion_de_arboles_binarios_completos}

\chapter{Posiciones en árboles binarios}
\entrada{Posiciones_en_arboles_binarios}

% \chapter{Posiciones en árboles binarios completos}
% \entrada{Posiciones_en_arboles_binarios_completos}
% 
% \chapter{Elemento del árbol binario completo según su posición}
% \entrada{Elemento_del_arbol_binario_completo_segun_su_posicion}
% 
% \chapter{Aproximación entre pi y e}
% \entrada{Aproximacion_entre_pi_y_e}

% ---------------------------------------------------------------------

\end{document}

%%% Local Variables: 
%%% mode: latex
%%% TeX-master: t
%%% End: 

