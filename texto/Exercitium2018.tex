% Exercitium2018.tex 
% Exercitium (curso 2018-19)
% José A. Alonso Jiménez
% Sevilla, 12 de enero de 2019
% ======================================================================

\documentclass[a4paper,12pt,twoside]{book}

%%%%%%%%%%%%%%%%%%%%%%%%%%%%%%%%%%%%%%%%%%%%%%%%%%%%%%%%%%%%%%%%%%%%%%%%
%% § Paquetes adicionales
%%%%%%%%%%%%%%%%%%%%%%%%%%%%%%%%%%%%%%%%%%%%%%%%%%%%%%%%%%%%%%%%%%%%%%%%

% Configuración para XeLaTeX
\usepackage{fontspec}
\usepackage{xltxtra}
\defaultfontfeatures{Ligatures=TeX,Numbers=OldStyle}
\setromanfont{DejaVu Sans}
% \setsansfont{Arial}
\setmonofont{DejaVu Sans Mono}[Scale={0.90}]

% Notas: La lista de fuentes disponibles se obtiene con fc-list

% \usepackage{ucs}
% \usepackage[utf8]{inputenc}        % Acentos de UTF8
\usepackage[spanish]{babel}        % Castellanización.
% \usepackage[T1]{fontenc}           % Codificación T1 con European Computer
%                                    % Modern.  
% \usepackage{graphicx}
% \usepackage{fancyvrb}              % Verbatim extendido
% \usepackage{mathpazo}              % Fuentes semejante a palatino
% \usepackage[scaled=.90]{helvet}
% \usepackage{cmtt}
% \renewcommand{\ttdefault}{cmtt}
\usepackage{a4wide}
\usepackage{minted}
\usepackage{comment}
\usepackage{amssymb, amsmath, amsbsy}

\usepackage{titletoc}
\dottedcontents{chapter}[0em]{}{24em}{1pc}
% \dottedcontents{chapter}[<left>]{<above-code>}{<label width>}{<leader width>}

\linespread{1.05}                  % Distancia entre líneas
\setlength{\parindent}{2em}        % Indentación de comienzo de párrafo
\setlength{\parskip}{1ex}          % Distancia entre párrafos
% \deactivatetilden                  % Elima uso de ~ para la eñe
\raggedbottom                      % No ajusta los espacios verticales.

\usepackage[%
  colorlinks=true,
  urlcolor=blue,
  % pdftex,
  pdfauthor={José A. Alonso <jalonso@us.es>},%
  pdftitle={Exercitium (curso 2018-19)},%
  pdfstartview=FitH,%
  bookmarks=false]{hyperref}      

\setcounter{tocdepth}{1}
\setcounter{secnumdepth}{4}

\usepackage{tocstyle}
\usetocstyle{KOMAlike}

% \usepackage{tocloft}
% \renewcommand\cftpartnumwidth{3cm}

% \usepackage{minitoc}

% \setlength\cftparskip{-2pt}
% \setlength\cftbeforechapskip{0pt}

\usepackage{epigraph} % Para citas al principio del capítulo

%%%%%%%%%%%%%%%%%%%%%%%%%%%%%%%%%%%%%%%%%%%%%%%%%%%%%%%%%%%%%%%%%%%%%%%%%%%%%%
%% § Cabeceras                                                              %%
%%%%%%%%%%%%%%%%%%%%%%%%%%%%%%%%%%%%%%%%%%%%%%%%%%%%%%%%%%%%%%%%%%%%%%%%%%%%%%

\usepackage{fancyhdr}

\addtolength{\headheight}{\baselineskip}

\pagestyle{fancy}

\cfoot{}

\fancyhead{}
\fancyhead[RE]{\mdseries\sffamily Exercitium (2018--19)}
\fancyhead[LO]{\mdseries\sffamily \nouppercase{\leftmark}}
\fancyhead[LE,RO]{\mdseries\sffamily \thepage}

%%%%%%%%%%%%%%%%%%%%%%%%%%%%%%%%%%%%%%%%%%%%%%%%%%%%%%%%%%%%%%%%%%%%%%%%
%% § Definiciones
%%%%%%%%%%%%%%%%%%%%%%%%%%%%%%%%%%%%%%%%%%%%%%%%%%%%%%%%%%%%%%%%%%%%%%%%

\input definiciones
\def\mtctitle{Contenido}

%%%%%%%%%%%%%%%%%%%%%%%%%%%%%%%%%%%%%%%%%%%%%%%%%%%%%%%%%%%%%%%%%%%%%%%%
%% § Título
%%%%%%%%%%%%%%%%%%%%%%%%%%%%%%%%%%%%%%%%%%%%%%%%%%%%%%%%%%%%%%%%%%%%%%%%

\title{
  {\LARGE Exercitium (curso 2018--19) \\
  {\Large Ejercicios de programación funcional con Haskell \\
  {\normalsize (hasta el 4 de enero de 2012)}}} }  
\author{\href{http://www.cs.us.es/~jalonso}
        {\Large José A. Alonso Jiménez}}
\date{\vfill \hrule \vspace*{2mm}
  \begin{tabular}{l}
      \href{http://www.cs.us.es/glc}
           {Grupo de Lógica Computacional} \\
      \href{http://www.cs.us.es}
           {Dpto. de Ciencias de la Computación e Inteligencia Artificial} \\
      \href{http://www.us.es}
           {Universidad de Sevilla}  \\
      Sevilla, \today % 12 de enero de 2019
  \end{tabular}\hfill\mbox{}}

%%%%%%%%%%%%%%%%%%%%%%%%%%%%%%%%%%%%%%%%%%%%%%%%%%%%%%%%%%%%%%%%%%%%%%%%
%% § Documento
%%%%%%%%%%%%%%%%%%%%%%%%%%%%%%%%%%%%%%%%%%%%%%%%%%%%%%%%%%%%%%%%%%%%%%%%

% \includeonly{Listas_equidigitales}

% \includexmp{licencia}

\begin{document}
% \dominitoc

\maketitle
\newpage

\input{licenciaCC}
\newpage

\newpage

\mbox{} \vspace*{2cm}
  \begin{verse}
  ``Sorpresas tiene la vida, \\
  Guiomar, del alma y del cuerpo; \\ 
  que nadie guarde hasta el fin \\
  el nombre que le pusieron; \\
  nadie crea ser quien dicen \\
  que es, ni que pueda serlo.'' \\ \vspace*{2ex}

  De Antonio Machado
  \end{verse}

\begin{flushright} 
\textit{Para Guiomar}
\end{flushright}

\newpage

\tableofcontents
\clearpage

\renewcommand{\chaptername}{Ejercicio}

\chapter*{Introducción}

% \mbox{} \hspace*{1cm} 

\begin{quote}
  ``\textit{The chief goal of my work as an educator and author is to
  help people learn to write beautiful programs.}''

  (Donald Knuth en
  \href{http://www.paulgraham.com/knuth.html}{Computer programming as an art})
\end{quote}

\vspace* {1cm}

Este libro es una recopilación de las soluciones de los ejercicios
propuestos en el blog
\href{https://www.glc.us.es/~jalonso/exercitium}
     {Exercitium}\
     \footnote{\url{https://www.glc.us.es/~jalonso/exercitium}}
durante el curso 2018--19.

El principal objetivo de Exercitium es servir de complemento a la
asignatura de
\href{https://www.cs.us.es/~jalonso/cursos/i1m-18}
     {Informática}\
     \footnote{\url{https://www.cs.us.es/~jalonso/cursos/i1m-18}}
de 1º del Grado en Matemáticas de la Universidad de Sevilla.

Con los problemas de Exercitium, a diferencias de los de las
\href{https://www.cs.us.es/~jalonso/cursos/i1m-18/ejercicios/ejercicios-I1M-2018.pdf}
     {relaciones}\
     \footnote{\url{https://www.cs.us.es/~jalonso/cursos/i1m-18/ejercicios/ejercicios-I1M-2018.pdf}},
se pretende practicar con los conocimientos adquiridos durante todo el
curso, mientras que con las relaciones están orientadas a los nuevos
conocimientos.

Habitualmente de cada ejercicio se muestra distintas soluciones y se
compara sus eficiencias.

La dinámica del blog es la siguiente: cada día, de lunes a viernes, se
propone un ejercicio para que los alumnos escriban distintas soluciones
en los comentarios. Pasado 7 días de la propuesta de cada ejercicio, se
cierra los comentarios y se publica una selección de sus soluciones.

Para conocer la cronología de los temas explicados se puede consultar el
\href{https://www.glc.us.es/~jalonso/vestigium/category/curso/i1m/i1m2018}
     {diario de clase}\
     \footnote{\url{https://www.glc.us.es/~jalonso/vestigium/category/curso/i1m/i1m2018}}.

En el libro se irán añadiendo semanalmente las soluciones de los
ejercicios del curso.

El código del libro se encuentra en
\href{https://github.com/jaalonso/Exercitium2018}
     {GitHub}\
     \footnote{\url{https://github.com/jaalonso/Exercitium2018}}

\section*{Cuaderno de bitácora}

En esta sección se registran los cambios realizados en las sucesivas
versiones del libro.

\subsection*{Versión del 16 de diciembre de 2018}

Se han añadido los ejercicios resueltos de la primera semana de
diciembre:

\begin{itemize}
\item \nameref{031218}
\item \nameref{041218}
\item \nameref{051218}
\item \nameref{061218}
\item \nameref{071218}
\end{itemize}
     
\subsection*{Versión del 22 de diciembre de 2018}

Se han añadido los ejercicios resueltos de la primera semana de
diciembre:

\begin{itemize}
\item \nameref{101218}
\item \nameref{111218}
\item \nameref{121218}
\item \nameref{131218}
\item \nameref{141218}
\end{itemize}
     
\subsection*{Versión del 29 de diciembre de 2018}

Se han añadido los ejercicios resueltos de la primera semana de
diciembre:

\begin{itemize}
\item \nameref{171218}
\item \nameref{181218}
\item \nameref{191218}
\item \nameref{201218}
\item \nameref{211218}
\end{itemize}
     
\subsection*{Versión del 29 de diciembre de 2018}

Se han añadido los ejercicios resueltos del 24 al 28 de
diciembre:

\begin{itemize}
\item \nameref{181224}
\item \nameref{181225}
\item \nameref{181226}
\item \nameref{181227}
\item \nameref{181228}
\end{itemize}

\subsection*{Versión del 12 de enero de 2019}

Se han añadido los ejercicios resueltos del 31 de diciembre al 4 de
enero: 

\begin{itemize}
\item \nameref{181231}
\item \nameref{190101}
\item \nameref{190102}
\item \nameref{190103}
\item \nameref{190104}
\end{itemize}

\chapter{Listas equidigitales}
\entrada{Listas_equidigitales}

\chapter{Distancia de Hamming}
\entrada{Distancia_de_Hamming}

\chapter{Último dígito no nulo del factorial}
\entrada{Ultimo_digito_no_nulo_del_factorial}

\chapter{Diferencia simétrica}
\entrada{Diferencia_simetrica}

\chapter{Números libres de cuadrados}
\entrada{Numeros_libres_de_cuadrados}

\chapter{Capicúas productos de dos números de dos dígitos}
\entrada{Capicuas_productos_de_dos_numeros_de_dos_digitos}

% ---------------------------------------------------------------------
 
\chapter{Números autodescriptivos}
\entrada{Numeros_autodescriptivos}

\chapter{Número de parejas}
\entrada{Numeros_de_parejas}

\chapter{Reconocimiento de particiones}
\entrada{Reconocimiento_de_particiones}

\chapter{Relación definida por una partición}
\entrada{Relacion_definida_por_una_particion}

\chapter{Ceros finales del factorial}
\entrada{Ceros_finales_del_factorial}

% ---------------------------------------------------------------------
 
\chapter{Números primos sumas de dos primos}
\entrada{Numeros_primos_sumas_de_dos_primos}

\chapter{Suma de inversos de potencias de cuatro}
\entrada{Suma_de_inversos_de_potencias_de_cuatro}

\chapter{Elemento solitario}
\entrada{Elemento_solitario}

\chapter{Números colinas}
\entrada{Numeros_colinas}

\chapter{Raíz cúbica entera}
\entrada{Raiz_cubica_entera}

% ---------------------------------------------------------------------

\chapter{Numeración de los árboles binarios completos}
\label{031218}
\entrada{Numeracion_de_arboles_binarios_completos}

\chapter{Posiciones en árboles binarios}
\label{041218}
\entrada{Posiciones_en_arboles_binarios}

\chapter{Posiciones en árboles binarios completos}
\label{051218}
\entrada{Posiciones_en_arboles_binarios_completos}

\chapter{Elemento del árbol binario completo según su
  posición}
\label{061218}
\entrada{Elemento_del_arbol_binario_completo_segun_su_posicion}

\chapter{Aproximación entre pi y e}
\label{071218}
\entrada{Aproximacion_entre_pi_y_e}

% ---------------------------------------------------------------------

\chapter{Menor contenedor de primos}
\label{101218}
\entrada{Menor_contenedor_de_primos}

\chapter{Árbol de computación de Fibonacci}
\label{111218}
\entrada{Arbol_de_computacion_de_Fibonacci}

\chapter{Entre dos conjuntos}
\label{121218}
\entrada{Entre_dos_conjuntos}

\chapter{Expresiones aritméticas generales}
\label{131218}
\entrada{Expresiones_aritméticas_generales}

\chapter{Superación de límites}
\label{141218}
\entrada{Superacion_de_limites}

% ---------------------------------------------------------------------

\chapter{Intercambio de la primera y última columna de
  una matriz}
\label{171218}
\entrada{Intercambio_de_la_primera_y_ultima_columna_de_una_matriz}

\chapter{Números primos de Pierpont}
\label{181218}
\entrada{Numeros_primos_de_Pierpont}

\chapter{Grado exponencial}
\label{191218}
\entrada{Grado_exponencial}

\chapter{Divisores propios maximales}
\label{201218}
\entrada{Divisores_propios_maximales}

\chapter{Árbol de divisores}
\label{211218}
\entrada{Arbol_de_divisores}

% ---------------------------------------------------------------------

\chapter{Divisores compuestos}
\label{181224}
\entrada{Divisores_compuestos}

\chapter{Número de divisores compuestos}
\label{181225}
\entrada{Numero_de_divisores_compuestos}

\chapter{Tablas de operaciones binarias}
\label{181226}
\entrada{Tablas_de_operaciones_binarias}

\chapter{Reconocimiento de conmutatividad}
\label{181227}
\entrada{Reconocimiento_de_conmutatividad}

\chapter{Árbol de subconjuntos}
\label{181228}
\entrada{Arbol_de_subconjuntos}

% ---------------------------------------------------------------------

\chapter{El teorema de Navidad de Fermat}
\label{181231}
\entrada{El_teorema_de_Navidad_de_Fermat}

\chapter{El 2019 es apocalíptico}
\label{190101}
\entrada{El_2019_es_apocaliptico}

\chapter{El 2019 es malvado}
\label{190102}
\entrada{El_2019_es_malvado}

\chapter{El 2019 es semiprimo}
\label{190103}
\entrada{El_2019_es_semiprimo}

\chapter{El 2019 es un número de la suerte}
\label{190104}
\entrada{El_2019_es_un_numero_de_la_suerte}

% ---------------------------------------------------------------------

\chapter{Cadena descendiente de subnúmeros}
\label{190107}
\entrada{Cadena_descendiente_de_subnumeros}

\chapter{Mínimo producto escalar}
\label{190108}
\entrada{Minimo_producto_escalar}

\chapter{Numeración de ternas de naturales}
\label{190109}
\entrada{Numeracion_de_ternas}

\chapter{Subárboles monovalorados}
\label{190110}
\entrada{Subarboles_monovalorados}

% \chapter{Mayor prefijo con suma acotada}
% \label{190111}
% \entrada{Mayor_prefijo_con_suma_acotada}

% ---------------------------------------------------------------------

\end{document}

%%% Local Variables: 
%%% mode: latex
%%% TeX-master: t
%%% End: 

